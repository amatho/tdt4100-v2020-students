\documentclass[12pt,a4paper]{article}

\title{Plan for Øving 4}
\author{}
\date{}

\begin{document}

\maketitle

\begin{itemize}
    \item Applikasjonen er en huskeliste ("todo list"), hvor man kan legge inn elementer med tittel og beskrivelse.
    \item Brukergrensesnittet består av to sider (tabs). Den første inneholder listen over nåværende
          huskeliste-elementer, mens den andre er for å legge til nye elementer til listen.
    \item Appen skal hovedsakelig bestå av to klasser (utenom "App"-klassen og kontrolleren):
          \begin{itemize}
              \item \textbf{Todo} - Dette er representasjonen av ett element i huskelisten. Her vil den overstyrte
                    "equals"-metoden være hovedmetoden. Denne metoden sammenligner UUID-ene til to Todo'er, for å
                    sjekke om de er like.
              \item \textbf{TodoListView} - Ansvarlig for å generere UI'en for selve huskelista. Hovedmetoden er
                    "updateView", som oppdaterer listen i brukergrensesnittet. Denne itererer over listen av Todo'er
                    og danner tilsvarende UI-elementer.
          \end{itemize}
    \item Lagringen til appen innebærer å lagre en representasjon av listen over Todo'er, slik at man kan beholde den
          samme listen over flere kjøringer av programmet.
\end{itemize}

\end{document}
